\section{Correcciones Bitácora 1}
Se realizaron las correcciones de aspecto y formato. Las correcciones más relevantes se detallan a continuación:

\begin{itemize}
    \item \textbf{Corrección 3.1} \\
    Para poder mezclar las dos ideas en una y enfocarnos más en la correlación, se decidió hacer uso del método delta, con el que se buscará aproximar la distribución de la variable aleatoria del Índice de Felicidad, por medio de las correlaciones obtenidas en los diferentes procesos dados. Se actualizan los consiguientes dados por las sugerencias siguientes en conjunto con la teoría y metodología del método delta. 
    
    \item \textbf{Corrección 5.1}\\
    Se identificaron nuevas tensiones en base a lo recomendado, ampliando a los posibles problemas con las bases de datos, encontrar un coeficiente de correlación adecuado, resultados contradictorios, etc.
    
    \item \textbf{Corrección 16.1} \\
    Se redujo la cantidad de variables a considerar para llevar a cabo el estudio de manera más centralizada. De esta manera, las variables con mayor relevancia son:

    \begin{itemize}
        \item Acceso a la electricidad. (electricity\_access)

        \item Acceso al agua potable. (water\_access)

        \item Clase social. (income\_class)

        \item Índice de Percepción de la Corrupción. (cpi)
        
        \item PIB per cápita (log\_gdp\_per\_capita)
    \end{itemize}
    
\end{itemize}

\section{Sugerencias Bitácora 1}
A partir del foro consideramos todas las sugerencias:

\begin{enumerate}
    \item \textit{En primer lugar, considero necesario reducir la cantidad de variables que se están contemplando para el desarrollo del trabajo. Sería prudente enfocarse únicamente en aquellas respaldadas por fuentes confiables. Esto permitiría contar con fundamentos sólidos y tener una idea más clara de los posibles resultados. Además, esta acción reduciría considerablemente el tiempo dedicado al trabajo, evitando posibles pérdidas de tiempo.} 

    \textbf{Respuesta:} Es cierto, aunque muchas variables, las ocupamos en esta bitácora para un análisis EDA. Usaremos y delimitaremos las más relevantes al estudio, como se hará próximamente. Recortar las variables no podíamos argumentarlo en la bitácora pasada porque todas se veían relacionadas directa o indirectamente, en el caso de esta podemos seccionar y delimitar claramente. Igual gracias por la sugerencia, la tomamos en esta bitácora. 

    \item \textit{Respecto a la explicación detallada de los aspectos de la UVE-Heurística, pienso que es innecesaria. La información proporcionada en la UVE debería ser suficiente para nuestros propósitos.}

    \textbf{Respuesta:} Se entiende la sugerencia, pero se considera como un borrador para la UVE. Totalmente recortable, pero era el propósito para la bitácora. Se puede eliminar perfectamente en el proyecto final, eso sí.

    \item \textit{En cuanto a los índices de cuadros y fichas, considero que no es necesario incluirlos al final del trabajo. Si decidimos mantenerlos, sugiero colocarlos en la primera página o simplemente eliminarlos por completo.}

    \textbf{Respuesta:} Se toma en cuenta la sugerencia para evitar abrumar al lector o recargar de índices la investigación. Se agradece la observación!
    
    \item \textit{Además, es necesario desarrollar las metodologías a utilizar. Ya que no se encuentran presentes en el trabajo, lo cual puede generar dudas sobre cómo se planea llevar a cabo el trabajo. Además, aunque las teorías planteadas no mencionan métodos específicos, en la parte escrita se menciona la correlación y algunos datos interesantes que podrían respaldar estas teorías. Sería pertinente hacer una mención de estas correlaciones en la sección de teorías y metodologías.} 

    \textbf{Respuesta:} Aunque fue la misma sección que se intenta obviar en una sugerencia anterior, pues sí sería pertinente ampliar las metodologías para una mayor comprensión general. Se realiza en conjunto con una sugerencia posterior.
    
    \item \textit{Algunos detalles observados en la parte escrita son que gran parte de las citas son textuales y utilizan el formato de parafraseo. En algunos casos, no se presentan las comillas de finalización, y algunos detalles con respecto a las comillas hacen que las letras al inicio de las comillas se eleven.}

    \textbf{Respuesta:} Al menos el problema de las comillas también era una de las correcciones principales, pero gracias por denotarlo!  

    \item \textit{Durante la lectura, encontré algunos errores ortográficos y mal uso de signos de puntuación. Además, muchas variables de la base de datos no se mencionan en la revisión bibliográfica. Me hubiese gustado tener una pincelada de información sobre por qué podrían ser relevantes esas variables en la investigación. Les recomiendo desarrollar más sobre los índices de libertad, contaminación, generosidad y entre otras, para no ponerle tanto peso solo a variables macroeconómicas durante la investigación. }

    \textbf{Respuesta:} Los errores ortográficos y puntuación los estamos corrigiendo, gracias por el enfoque! Por otro lado, la idea del estudio es comparar ambos, los factores económicos y sociales (felicidad) de forma general, pero podríamos enfocar un poco en las diferentes relaciones como se menciona. Además, en la revisión bibliográfica con este mismo enfoque buscamos artículos de forma general que tratasen ambos, para ver diferentes perspectivas del estudio. Tal vez a futuro si logramos encontrar una relación muy fuerte buscaremos literatura para poder enlazarla mejor.

    \item \textit{Sería bueno que arreglaran los problemas de los cuadros rojos en el documento}

    \textbf{Respuesta: } Gracias al aporte del compañero y el apoyo personal del mismo, logramos quitar los cuadros rojos gracias al subpaquete de \textit{hyperref} llamado \textit{hidelinks}
    
    \item \textit{Lo segundo por mejorar sería una mejor identificación de las tensiones, pues han dejado varias cosas por fuera como por ejemplo: el acceso a la desigualdad a la atención médica, la inseguridad alimentaria, desplazamientos forzados y refugiados (ejemplo de ellos los inmigrantes), también como afecta el cambio climático y la degradación ambiental (sequías, escasez de agua), violencia de género y discriminación, marginalización de grupos étnicos y minorías (como el caso Israel - Palestina), los desafíos tecnológicos como por ejemplo la automatización de trabajos y las brechas digitales. También revisar muy bien lo escrito pues hay algunos errores ortográficos.}

    \textbf{Respuesta:} Es una buena recomendación de los compañeros, aunque sentimos que lo habíamos generalizado un poco para no entrar en tantos casos específicos, pero puede expandirse un poco el tema, además de que las tensiones tuvieron que ser corregidas. Gracias igual! Los errores ortográficos los pasamos viendo igual. 

    \item \textit{En cuanto a la redacción, algunas partes del texto son un poco redundantes o pueden expresar la idea sin usar tantas palabras. Además, es importante incluir comillas al realizar citas textuales y recordar que las citas en bloque no deben sobrepasar las 40 palabras según las normas APA 7.}

    \textbf{Respuesta:} Gracias por la observación, los errores al citar se corrigieron inmediatamente. Consideraremos evitar redundancias para las siguientes entregas.

    \item \textit{Como pequeño detalle de escritura, vi varias U escritas con diéresis. Por otra parte, del trabajo, entiendo que quieren encontrar la distribución del índice de felicidad para comprender mejor su comportamiento e intentar obtener conclusiones de eso, sin embargo, siento que deben ampliar un poco más en como harán eso, procedimentalmente, que aplicarán, que técnica o método, quizás también incluir más bibliografía "matemática".}

    \textbf{Respuesta:} Lo de las U fueron por las comillas. Por otro lado, en consideración con las correcciones, decidimos enfocar el tema para poder ampliar en el aspecto, y así lograr una mayor expansión matemática. Gracias! No lo habíamos tenido en mente, entonces podemos expandir un poco ese aspecto.
    
    \item \textit{Como objeto de estudio determinaron como distintos indicadores de progreso socioeconómico se relaciona con el índice de felicidad. El hecho de que tomaran múltiples indicadores y no solo unos cuantos, puede ser complicado de sobrellevar dada la gran cantidad de análisis que se pueden realizar. Me parece que la delimitación de los indicadores podría brindar mayores ventajas en etapas posteriores de la investigación.} 

    \textbf{Respuesta:} Justamente la delimitación la haremos en esta bitácora, para poder hacer la situación más manejable y recomendable, claramente el estudio intenta comparar los factores económicos y progresos sociales, entonces delimitar desde antes sería cortar ramas de resultados. Pero gracias, tenemos en cuenta la cantidad de variables y por eso mismo se delimitan a continuación. 
    
\end{enumerate}

En general, las sugerencias apuntan hacia tres sentidos: detalles de citas y ortografía; expansión de teoría y metodología; y delimitación de las variables. Consideramos que lograremos acatar las sugerencias y correcciones hasta acá y se les agradece mucho el aporte. 

