\section{Correcciones Bitácora 3}
Se hicieron las correcciones de estructura y bibliografía, al menos para esta bitácora no se recomendaron cambios extraordinarios.

\section{Sugerencias Bitácora 3}

\begin{enumerate}

    \item \textit{Hubiese sido muy útil una explicación sobre los cambios que realizaron en el código para la bitácora. Se evidencia un desarrollo de la programación en la bitácora 2 y en el GitHub pero a la hora de seguirle la pista a las mejorías hubiese sido de gran ayuda una explicación de las nuevas líneas de código que programaron.}\\

    \textbf{Respuesta:} A nivel general, se optó por incluir los gráficos y las tablas generadas más que el código tal, esto con el fin de concentrar mejor la información y que cualquier persona lo pueda entender. De igual manera, se dio énfasis a la explicación de las metodologías aplicadas, ya que estas son la esencia del código. ¡Se agradece la observación!
    
    \item \textit{Mis comentarios son de la parte estética, pues en el apartado 3.2 de la parte de escritura, considero que tuvieron un pequeño fallo pues al realizar la cita del Instituto Nacional del Cáncer, escribieron primero el autor, unas palabras más y después la cita, creo que lo usual y más estético es colocar la cita e inmediatamente después el autor, también en el apartado que hablan sobre la transformación Z de Fisher hay un pequeño defecto visual, pues el logaritmo tiene paréntesis que no se ajustan adecuadamente al contenido esto lo pueden arreglar utilizando \textbackslash{}left( \textbackslash{}right).}\\

    \textbf{Respuesta:} En efecto, estas fueron parte de las correcciones que realizamos anteriormente, pero de igual modo, ¡Gracias por enfatizarlo!

    \item \textit{Me cuesta encontrar algo para mejorar su investigación, realmente la encuentro muy completa, tal vez me parece extraño agregar el código durante la bitácora 2 y luego en la bitácora 3 no seguir ese formato de agregar el código que utilizan. En lo personal no hubiese agregado código en ningún momento, creo que no es particularmente valioso agregar el código, en especial agregar cosas como al ggsactter o el ggplot, no le aporta mucho al lector.}\\

    \textbf{Respuesta: }De hecho, esa fue la razón por la que se decidió eliminar por completo el código. En un inicio, en la bitácora 2 se agrego explícitamente el código ya que no se tenían los resultados directamente. Por otro lado, para la bitácora 3 se decidió hacer uso de gráficos y tablas ya que son de más agrado para cualquier persona. ¡Gracias por la observación!
    
\end{enumerate}



