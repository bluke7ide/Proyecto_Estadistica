\section{Correcciones Bitácora 2}
Se realizaron las correcciones de aspecto y formato, además se mejoró la ortografía junto a la redacción. Las correcciones más relevantes se detallan a continuación:

\begin{itemize}
    \item \textbf{Corrección 24.1: Con respecto a conectores de texto.} \\
    Se revisó detenidamente el documento y se le dio mayor unicidad por medio de conectores de texto, de forma en que se logró mejorar la naturalidad a la hora de leer cada parte. 
    
    \item \textbf{Corrección 34.1, 34.2, 37.1: Con respecto a tablas} \\
    Se formalizaron las tablas, redondearon números y se mejoraron en sentido visual para el lector, aedmás se cambiaron los puntos decimales por coma como lo sugerido, aunque realizando un poco de investigación en el documento sugería que se podían usar ambos. 
    
    \item \textbf{Corrección 35.1, 35.2, 38.2, 39.1, 39.2: Con respecto a gráficos} \\
    Se utilizó la sugerencia de exportar los gráficos en formato pdf en vez de imágenes, de esta manera se logra que la calidad se encuentre siempre al 100\%, independientemente del zoom que se haga. Por otro lado, se nos había olvidado el comando theme\_minimal() en algunos, y en general volvimos a hacer todos los gráficos para una mayor representación visual. 
\end{itemize}

\section{Sugerencias Bitácora 2}

\begin{enumerate}
    \item \textit{Los cuadros en donde agregan código de R son claramente relevantes, sin embargo, en mi opinión no tienen por qué aparecer directamente en la bitácora, en especial cuando generan los gráficos, visualmente siento que el código no aporta mucho al trabajo (en el sentido visual, claro que los gráficos son muy importantes), al dejar los gráficos respectivos, sin el código explícitamente, considero que se solucionaría este problema.}

    \textbf{Respuesta:} Los incluimos de manera para que se pudiera ver un pequeño procedimiento de ambos métodos. Aunque claramente como nos indica la sugerencia no aporta mucho al trabajo, sentimos dejarlos como están puesto son un pequeño ejemplo. Aunque para el trabajo final, si los consideramos remover puesto lo mismo, que consideró la sugerencia. 

    \item \textit{Por otro lado, no encontré el repositorio de github en el documento, desconozco si no tenían uno para este momento, pero pienso que sería bueno agregarlo en un anexo o apéndice para ver el código que han realizado hasta el momento.}

    \textbf{Respuesta:} Ya agregamos el repositorio de GitHub, tanto en la bitácora 2, como en la 3, y también lo agregamos a esta sección, por si acaso. ¡Gracias! Se nos había olvidado agregarlo.

    \item \textit{Con toda sinceridad me costó encontrar que escribir acá ya que de verdad la bitácora me parece muy buena, algo que noté es que algunas variables del dataset tienen nombres un poco extensos, quizás se podría pensar en una abreviación, claro que sea representativa, con el fin de hacer los nombres más cortos.}

    \textbf{Respuesta:} Claro, eso nos afectó en las tablas y lo resolvimos con abreviarlas, aunque no cambiarles el nombre. Podríamos considerar modificarlos si nos causan más problemas. ¡Gracias!

    \item \textit{No queda claro que aporte o papel cumplen las Figuras 2.5, 2.6 y 2.7. Hace falta explicar cuál es el propósito e importancia de las mismas, así como su interpretación. Podrían explicar que representa o que mide exactamente su variable de interés "life ladder" y cómo esto se relaciona con su objetivo de investigación.}

    \textbf{Respuesta:} En general, corregimos los gráficos para que sean más fáciles de entender para el lector, de igual manera, se unificaron con el texto de manera en que antes de llegar al gráfico, haya una breve explicación del mismo. Por otro lado, se procurará mejorar la explicación de la variable a estudiar para las siguientes entregas. ¡Gracias!

    \item \textit{Las tablas (página 31) presentadas en el trabajo no tienen un formato que visualmente le sea agradable al lector.  Además, no cumplen con las normas de presentación estadística requerida:  https://admin.inec.cr/sites/default/files/media/mepresentinfoestadist-21122017\_2.pdf. El gráfico de la figura 2.2 pueden explicarlo un poco más en el párrafo, dado que cuesta entender lo que se quiere mostrar, no tiene una "visualización clara" como lo mencionan. No encontré en su trabajo el link del repositorio del GITHUB  que se pedía en la bitácora.}

    \textbf{Respuesta:} ¡Se agradecen las observaciones! Como se mencionó anteriormente, todas las tablas fueron mejoradas para lograr un mejor entendimiento para el lector, y que este a su vez sea más agradable. De igual forma, los gráficos se pulieron y se incluyó explicación de los mismos en el texto. Por último, se agregó el enlace al repositorio ya que este si fue un descuido que olvidamos.
    
\end{enumerate}